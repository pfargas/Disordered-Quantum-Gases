\subsection{Anderson localization}

In 1958, P. W. Anderson \cite{andersonAbsenceDiffusionCertain1958} showed that enough structural disorder in a solid-state system, can exhibit localisation of states. Those states present an exponential decaying tail in the probability density function, which leads to the confinement of particles in times much larger than the characteristic time of the system which implies that quantum particles inside such a system can be confined in a region of space in which classical particles are not.

This phenomena emerges when treating particles as waves, and studying the scattering processes that take place. If the disorder is strong enough, the wave function interacts with itself resulting in a constructive interference in a small region of space.

This effect has been observed in various systems, such as light in disordered media \cite{wiersma_localization_1997} or sound waves in acoustic systems \cite{hu_localization_2018}. In the case of electrons, Anderson localisation is a fundamental concept in condensed matter physics, as it is expected to be responsible for the insulating behaviour of disordered materials. Understanding this effect is crucial for the development of the knowledge in electronic transport properties.

% Examples in russian video

\subsection{Ultracold atoms for quantum simulation}

Ultracold atoms have been demostrated to be a versatile platform for the study of quantum many-body systems. The ability to control the interactions between atoms [Feshbach resonances], the external potential [laser physics], and the temperature of the system [evaporative cooling...], allows for the study of a wide range of phenomena and interactions, such as superfluidity, bose-einstein condensation, dipolar systems, and many others.

In 2015, Gavish and Castin \cite{gavishMatterwave2005} proposed this platform as a way to study Anderson localisation for a matter wave. The proposal was to build an optical lattice with a low filling factor of an atom B, the scatterers, such that the effect of the lattice would not be noticeable. Then, a second species of atoms A, would be released into this disordered media with a low energy, such that the interaction between A and B would be an s-wave ($l=0$ in the angular momentum). The scatterers are prepared in the vibrational ground state of the optical lattice, and the atoms A interact with B atoms with a lower energy than the recoil energy of the lattice, such that the interaction is elastic (B atoms don't move). The first approximation to model this kind of s-wave interaction is a delta potential, which is isotropic, meaning that the particles B are modeled as infinitessimally thin scatterers.

\subsection{The pseudopotential approximation. Conditions in the wave function}

As stated before, in the setup considered in this work, the interaction between the scatterers and the matter wave is modelled as the typical Yuan and Hang 2D regularized delta-like pseudopotential. 

The pseudopotential technique consists in approximating a potential with some other easier potential, given some conditions. In our case, we consider that the B particles are far away from each other, so that the overlap of the wavefunction between B atoms is negligible. For the low filling factor considered, this should be a good approximation. The regularized delta pseudopotential is given by \cite{farrellSwave2010}:

\begin{equation}
    V(\mathbf{r}) = \frac{\pi\hbar^2}{m}\delta(\vb{r})\qty(1-r\ln{\qty(\frac{r}{2ae^{1-\gamma}})}\partial_{r})
\end{equation}

This potential is a first approximation to the problem. We should take into account that this picture is not the full picture of the problem, for example, as this potential is infinitessimally thin, we can't study percolation processes in this system, which could lead to localization of states due to an actual barrier, while with gaussians, if the filling factor is high enough, and the B atoms extense enough, this effect could have a big impact.

The description of the system with this potential, can also be understood in terms of the free Schrödinger equation, with boundary conditions at the position of the scatterers. The Wigner-Bethe-Peierls condition is an equivalent treatment of the problem with this pseudopotential \cite{betheQuantum1997} but it has a nicer interpretation. The condition in 2D by a set of scatterers, whose positions are the set $\{\vb{r}_i\}$ is described as follows:

\begin{equation} % PUT CONDITION
    \psi(\vb{r}) = \frac{m}{\pi\hbar ^2}D_i\ln{\qty(\frac{|\vb{r}-\vb{r}_i|}{a_{eff}})}+O(|\vb{r}-\vb{r}_i|)
\end{equation}

Which means that we will describe the problem as free particles with these boundary conditions.

\subsection{The Fermi gas: Non-interacting fermions}

The final objective of this thesis is to study a non-interacting Fermi sea in the presence of disordered scatterers in a finite volume. The Fermi sea is a quantum state of a system of fermions, in which, due to Pauli exclusioin principle, the fermions occupy the lowest energy states available up to a threshold $E_F$: the Fermi energy $E_F$, which is defined as the energy of the particle at the Fermi level, that is, the energy of the highest occupied state, and it is strongly related with the number of particles $\mathcal N$ in the system.

This means that a Fermi sea can be described as $\mathcal{N}$ free (therefore independent) particles, with the energies allowed by the Pauli exclusion principle. 

The Fermi sea can be a good model to study the transport properties of a system at low temperatures. If we suppose that a system is correctly described by a Fermi sea, the system should be conductor, as the eigenstates of the free particle hamiltonian is an extended state. If we wanted to explain the insulating behaviour of a system, we should consider the interaction between the particles, leading to the usual Fermi-Hubbard model.

But in our case, as Anderson localization is a single particle phenomena, we can have a Fermi sea of non-interacting particles, and even so, find that the system is insulating.
\color{blue}
{To extract the spectrum of the fermi sea in presence of the scatterers, we model the scatterers as very sharp gaussians, with a width $\sigma = d/4$. The random potential is given by:

\begin{equation}
    V(\vb{r}) = \sum_i V_i(\vb{r}) = \sum_i V_0\exp{\qty(-\frac{|x-{r}^x_i|^2+|y-{r}^y_i|^2}{2\sigma^2})}
\end{equation}

Where $V_0$ is chosen such that:

\begin{equation}
    \int g\delta(\vb{r})d^2 \vb{r}=\int V(\vb{r})d^2 \vb{r}
\end{equation}

Which yields $V_0 = \frac{g}{2\pi\sigma^2}$. Then, we can diagonalize the hamiltonian.}
\color{black}