\clearpage
\newpage
\section{Introduction}

The \textit{wave-particle duality} is one of the most famous behaviours that arise when considering particles in the quantum regime. The Schrödinger equation treats quantum objects as a complex wave, \textit{the wavefunction of the particle}, the physical interpretation of which is that the square of this wavefunction represents the probability distribution function of the position of the quantum object.

This new way of understanding particles makes wave phenomena possible, like interference when two particles interact with each other or diffusion in a medium. In this work, we try to understand a wave phenomenon called \textit{localisation}. 

In the classical sense, all particles are localized: they have a defined position, but for a wave, this statement might not be true: In general, waves span over a large region, for example, if we emit an electromagnetic wave from a source into an empty universe this wave will be propagated throughout all space. Using the interference of different waves, one can also find localised waves, such as the \textit{wave packet}.

The aim of this work is to characterize the localization of a Fermi gas in a disordered media of point-like scatterers, searching how many states are localized given a scattering length $a_{eff}$ and a Fermi energy $E_F$. The system is modelled as a set of point-like scatterers in a 2D square lattice, and the particles are non-interacting fermions in a box in the disorder with periodic boundary conditions. 

\subsection{Localization phenomena in condensed matter physics}

Arguably, the most known model for lattice systems is the Hubbard model. In it, the particles are described by the interplay between the kinetic and interaction energies. The Hubbard model is a simplified model that can describe the electronic structure of solids, where the electrons are confined to a lattice of atoms. The electrons can hop between the lattice sites and interact with each other when they are in the same site. If the interaction energy is much larger than the kinetic energy, the electrons will be localized in the lattice sites and the system will be an insulator. This phase is called the Mott insulator phase and the localized character is a many-body effect.  

However, in disordered systems, the backscattering effects of a single particle with the disorder, can lead to a constructive interference of the wavefunction in some region of space, reducing the conductivity of the material in case of solid state materials. This is called weak localization and it is considered the precursor of the Anderson localization or strong localization phenomena. RANDOM WALK?



{An Introduction that clearly states the rationale of the thesis that includes:}

\begin{enumerate}
\item {Statement of purpose (objectives).}
\item {Requirements and specifications.}
\item {Methods and procedures, citing if this work is a continuation of another project or it uses applications, algorithms,
software or hardware previously developed by other authors.}
\item {Work plan with tasks, milestones and a Gantt diagram.}
\item {Description of the deviations from the initial plan and incidences that may have occurred. }
\end{enumerate}

\bigskip

{The minimum chapters that this thesis document should have are described below, nevertheless they can have different
names and more chapters can be added.}

\bigskip

\subsection{Gantt Diagram}
\label{ssec:gantt}
\begin{figure}[H]
    \centering
    %\includegraphics[width=13cm]{img/diagram_gantt.png}
    \begin{ganttchart}[y unit title=0.4cm,
y unit chart=0.5cm,
vgrid,hgrid,
title height=1,
today=25,%
today offset=.5,%
today label=Now,%
bar/.style={draw,fill=cyan},
bar incomplete/.append style={fill=yellow!50},
bar height=0.7]{1}{25}

 % dies
 \gantttitle{Phases of the Project}{25} \\
 \gantttitle{February}{5}
 \gantttitle{March}{5}
 \gantttitle{April}{5}
 \gantttitle{May}{5}
 \gantttitle{June}{5}\\
 
 % caixes elem0 .. elem9 
 \ganttgroup[inline=false]{Theory}{1}{15}\\
 \ganttbar[progress=100]{Read papers}{1}{6} \\
 \ganttbar[progress=100]{Redo calculations}{5}{15} \\
 \ganttbar[progress=100]{Understand theory}{4}{16} \\
 \ganttgroup[inline=false]{Implementation}{3}{22}\\
 \ganttbar[progress=100]{Reproduce}{3}{10} \\
 \ganttbar[progress=100]{Optimize}{10}{20} \\
 \ganttbar[progress=100]{Analyse}{20}{22} \\
 \ganttgroup[inline=false]{Writting}{19}{25}\\
 \ganttbar[progress=100]{Thesis}{19}{25} \\
 
 % relacions
 \ganttlink[link type=f-f]{elem3}{elem6}
 \ganttlink[link type=f-f]{elem6}{elem7}
 \ganttlink[link type=s-s]{elem1}{elem3}
 \ganttlink[link type=f-f]{elem7}{elem9}
%  \ganttlink{elem2}{elem7}
%  \ganttlink{elem3}{elem7}
%  \ganttlink{elem5}{elem6}

\end{ganttchart}

    \caption[Project's Gantt diagram]{\footnotesize{Gantt diagram of the project}}
    \label{fig:gantt}
    For more information read the manual \cite{skalagantt} of Skala.
\end{figure}

\bigskip

\subsection{Topic}
