

\subsection{Framework used in this work}

This work has followed the mathematical formalism started in [Matterwave localisation in disordered cold atom lattices] and followed by [Massignan 2006] and [Quantitative study of two...]. 

The main object is the Green's function (or propagator) of the system. This object can be suitable for studying the transport properties of systems, as it can be used to solve the Schrödinger equation for a given potential. The Green's function is defined as the inverse of the operator $E-H_0$, where $H_0$ is the Hamiltonian of the system and $E$ is the energy of the system (see section []). 

\subsection{Criterium for localization}

The usual criterium for determining that some state is localized is the exponential decay of the wave function. This can be seen in the Green's function, as the Green's function is the propagator of the system, and it can be used to determine the wave function of the system.

But, one can see that this criterium could be not enough. Even though it is true that Anderson localization implies an exponential decay of the wave function, there is another situation that leads to this decay: if we inject particles in a spectral gap of the system, the wave function will decay exponentially. This is not Anderson localization, but a consequence of the spectral gap [reference exponential decay of correlation function in a spectral gap].

\subsection{Scaling theory: 2D as a critical dimension}

Briefly discuss the scaling theory for the conductivity. 3D mobility edge and 1D always localized. [Anderson and QFT condensed matter book]

\subsection{Method for effective calculation of M matrix}

Paralellise M matrix as follows:

\begin{itemize}
    \item M matrix is symmetric, compute only upper triangle without diagonal.
    \item The upper triangle is divided in 4 blocks
    \item Each block is a process
    \item then we add the 4 blocks, add transpose and add diagonal (separate process)
    \item Diagonal is not difficult (just a constant)
\end{itemize}

FINALLY, NUMBA CALCULATIONS. WAY EASIER.

\subsection{Numerical methods: Diagonalisation of M matrix, SVD and simulations}

DECOMPOSE THE M MATRIX (SVD or DIAGONALIZATION??), PYTHON AND METHODS USED. EFFICIENCY OF THE CODE.

\begin{equation}
    M  = U \Sigma V^T
\end{equation}

Where $U$ and $V$ are orthogonal matrices and $\Sigma$ is a diagonal matrix. The diagonal elements of $\Sigma$ are the singular values of $M$. The singular values are the square roots of the eigenvalues of $M^T M$ only if $M$ is a square matrix with positive eigenvalues (?).

Should I define $E=k^2$ or $E=\frac{k^2}{2}$?!!

NEWTON METHOD TO FIND RESONANCES:

\begin{itemize}
    \item Given a lattice of dispersors, set an energy $E>0$ and diagonalize $M$ such that you have a set of eigenvalues $m_i$.
    \item Compute $a_{eff}$ such that the real part of $m_i$, a given eigenvalue of $M$, is zero.
    \item $a_{eff}$ and $E$ are the axis of the histogram.
    \item Compute first step in Newton's method:
    \begin{equation}
        z_{res}=E \underbrace{-\text{i}\frac{\text{Im}(m^{\infty}_i)}{\frac{d m_i^{\infty}}{dE}}}_{\text{newton\_step in the code}}
    \end{equation}
    \item Hellmann-Feynman theorem to compute the derivative.
    \item With the expression $z_{res}=E_{res}-\text{i}\hbar \Gamma/2$ check that $\Gamma$ is small enough.
\end{itemize}